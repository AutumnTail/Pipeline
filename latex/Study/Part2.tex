\subsection{Identification of the dynamic response genes (DRG)}
\label{section:identification_of_drgs}

Let us assume that the centered expression profile of the $i^{th}$ gene, belonging to subject $j$, $X_{i,j}(t)$, is a smooth function over time and that the time course gene expression measurements are discrete observations from this smooth function, which have been distorted by noise, i.e., $Y_{i,j}(t_{k}) -\mu_{i,j} =X_{i,j}(t_{k})+\epsilon_{i,j}(t_{k})$, for $i=1,\ldots,n,$ $j=1,\ldots,N$ and $k=1,\ldots,K_{i,j},$ where $n$ is the number of genes, $N$ is the number of subjects, and $K_{i,j}$ is the number of time points observed for the $i^{th}$ gene, belonging to subject $j$. The noise denoted by $\epsilon_{i,j}(t_{k})$ is assumed to be an independently identically distributed (i.i.d.) random variable with mean $0$ and variance $\sigma^{2}$. The pipeline (Carey et al., 2016) obtains the functional entity $X_{i,j}(t)$ by spline smoothing \citep{green1993nonparametric,silverman2005functional}. Spline smoothing involves minimizing the following penalized residual sum of squares 

\begin{equation}
\label{eq:res_sum_squares}
(Y_{i,j}(t_{k}) -\mu_{i,j} - X_{i,j}(t))^{2} + \lambda \int \left[D^{2} X_{i,j}(t) \textrm{d}t\right]^{2}.
\end{equation}

The first term defines the discrepancy between the observed centred gene expression measurements $Y_{i,j}(t_{k}) -\mu_{i,j}$ and the functional entity $X_{i,j}(t),$ while the second term requires $X_{i,j}(t)$ to be sufficiently smooth. The parameter $\lambda$ controls the trade-off between these two competing effects and hence ensures that $X_{i,j}(t)$ has an appropriate amount of regularity or smoothness. As smoothing each individual gene is computationally expensive and is likely to obtain biased estimates as our data does not contain replicates and is expected to contain a low signal-to-noise ratio. The pipeline pools the genes together in order to obtain an estimate of the amount of regularity required. This is an approach similar to that of Yao et al. (2005) and Wu and Wu (2013).

As the majority of the genes have a relatively flat expression profile over time estimating the smoothing parameter by pooling all the genes together is not ideal as it will tend to over smooth the data. Therefore, the pipeline chooses a subset of the genes that exhibit time course patters that have relatively smooth trajectories that do not fluctuate widely. Then these genes are ranked by their interquartile range and select the top genes for the estimation subset. Figure~\ref{fig:smoothexp} illustrates the genes expresion curves in the estimation subset over time for each subject. 

The majority of the genes have a relatively flat expression levels over time thus estimating the regularity parameter $\lambda$ using all of the genes together is not ideal as it will tend select a $\lambda$ that is too large in order to minimize the prediction error for the majority of the unresponsive genes. As we are interested in obtaining an appropriate amount of regularity for the responsive genes we apply an approach similar to that of \cite{yao2005functional} and \cite{wu2013more} we choose a subset of the genes that exhibit time course patterns that have relatively smooth trajectories that do not fluctuate widely. Then we rank these genes by their interquartile range and select the top genes for our subset. The regularity parameter is estimated by minimizing the generalized cross validation (the prediction error) of the responsive genes in our estimation subset.

The smoothing parameter is estimated by the conventional generalized cross validation approach using the genes in our estimation subset. Once the smoothing parameter is selected one can proceed to the estimate the functional entity $X_{i,j}(t)$ using all gene expression profiles. Table 1 provides the optimal smoothing parameters, the corresponding degrees of freedom, the generalized cross validation and the sum of squared errors for the fitted curves $X_{i,j}(t)$ for each subject in the study respectively.

\subsection{Hypothesis testing to detect the dynamic response genes (DRG)}

In a typical gene expression experiment, tens of thousands of genes are measured simultaneously, but only a fraction of them are associated with the biological process of interest or a particular stimulus, such as a therapeutic treatment or virus infection. Since it is reasonable to include only these responsive genes in our analysis, here we identify dynamic response genes (DRGs), i.e., genes with expressions that have changed significantly over time.

The pipeline uses the following hypothesis to identify the DRGs,


\begin{itemize}

\item $H_{0}$: $X_{i,j}(t)$ equal to $0$

\item $H_{a}$: $X_{i,j}(t)$ not equal to $0$.

\end{itemize}

The test statistic is the conventional F-statistic, which compares the goodness-of-fit of the null model to the alternative model given by

\begin{equation}
\label{eq:altmodel}
F_{i,j}=\frac{RSS_{i,j}^0-RSS_{i,j}^1}{RSS_{i,j}^1},
\end{equation}

where 

\begin{equation} 
RSS_{i,j}^0= (Y_{i,j}(t_{k}) -\mu_{i,j} )^{2}
\end{equation}

and

\begin{equation} RSS_{i,j}^1=(Y_{i,j}(t_{k}) -\mu_{i,j} - X_{i,j}(t))^{2} \end{equation}

are the residual sum of squares under the null and the alternative models for the $i$-th gene, belonging to subject $j$. The genes with large F-ratios can be considered as exhibiting notable changes with respect to time. As we wish to have an equal amount of DRGs for each subject we rank the F-ratios and select $3000$ of the top ranking genes (TRGs).

\subsection{Clustering the DRGs into temporal gene response modules (GRMs)}
\label{section:identification_of_grms}

As many of the DRGs exhibit similar expression patterns over time we wish to cluster them into temporal gene response modules (groups of genes which have similar gene expression profiles over time). In order to achieve this we adopt the Iterative Hierarchal Clustering (IHC) method \citep{careycluster}. The IHC algorithm can identify inhomogeneous clusters, capture both the large and very small clusters and provides an automated selection of the optimal number of clusters. The IHC algorithm is only reliant on a single parameter, $\alpha$, which controls the trade-off for the between- and within-cluster correlations and is outlined below:

\begin{enumerate}
   \item Initialization: Cluster the gene expression curves using hierarchical clustering. Let the distance metric be the Spearman correlation with a threshold of $\alpha$.
   \item Merge: Treat each of the cluster centres (exemplars) as `new genes', use the same rule as in Step 1 to merge the exemplars into new clusters.
   \item Prune: Let $c_{i}$ be the centre of cluster $i$. If the correlation between the cluster centre and gene $j$, which will be denoted by $\rho_{i,j},$ is less than $\alpha$ remove $gene_{j}$ from the cluster $i$. Let $M$ be the number of genes removed from the existing $N$ clusters. Assign all $M$ genes into single-clusters. Hence there is now $(N+M)$ clusters in total.
   \item Repeat Steps 2 and 3 until either the index of clusters fully or partially converges or the number of iterations reaches the maximum number of iterations.
   \item Repeat Step 2 until the between-cluster correlations are less than $\alpha$.
\end{enumerate}

\subsection{Discovery of the high-dimensional gene regulation network (GRN) using differential equation models}
\label{section:identification_of_grn}

There has been an abundance of literature regarding the use of ordinary differential equation (ODE) modelling for constructing high-dimensional gene regulation networks (GRN) e.g., \cite{hecker2009gene}, \cite{lu2011high} and \cite{wu2013high}. A gene regulatory network, attempts to map how different genes control the expression of other genes. The gene regulations can be modelled by rate equations, $$ D M_{q,j} = \alpha_{0,q,j} + \sum_{p=1}^{Q} \alpha_{p,q,j}M_{p,j}, \qquad \textrm{for} \quad q =1,\ldots,Q, $$ where $\alpha_{0,q,j}$ is the intercept for the $q^{th}$ gene response module, belonging to subject $j$ and the coefficients $\{\alpha_{p,q,j}\}_{p=1}^{Q}$ quantify the regulation effects of the $p^{th}$ gene response module on the instantaneous rate of change in $q^{th}$ gene response module. This model can appropriately capture both up and down regulations as well as up and down self-regulations. Typically, only a few gene response modules will effect the instantaneous rate of change in $q^{th}$ gene response module thus only a few of the $\{\alpha_{p,q,j}\}_{p=1}^{Q}$ will be non-zero. We first perform a model selection which determines which $\{\alpha_{p,q,j}\}_{p=1}^{Q}$ are non-zero and then we estimate their coefficients to determine the regulation effects. Here we use the ordinary differential equation (ODE) modelling approach in order to reconstruct the high-dimensional gene regulation networks (GRN). In ODE network models, gene regulations are modeled by rate equations, which quantify the rate of change (derivative) of the expression level of one temporal gene response module in the system as a function of expression levels of all related temporal gene response modules. Both up and down regulations as well as self-regulations can be appropriately captured by the ODE model. The general form of the ODE model can be written as

\begin{equation} \frac{dM_{q,j}}{dt} = \alpha_{0,q,j} + \sum_{p=1}^{P} \alpha_{p,q,j}M_{p,j}, \end{equation}

for $i =q,..,Q$ where $M_{q,j}$ is the $q^{th}$ gene response module, belonging to subject $j$, $\alpha_{0,q,j}$ is the intercept for the $q^{th}$ gene response module, belonging to subject $j$ and the coefficients $\{\alpha_{p,q,j}\}_{i=1}^{P}$ quantify the regulation effects of other gene response modules on the rate of expression change of the q-th gene response module, belonging to subject $j$.

Although the model size Q is comparably smaller than that of the original model N, simultaneous model selection and parameter optimization of ODE parameters $\{\alpha{p,q,j}\}_{q,p=1}^{Q,P}$ are still computationally very expensive, because it involves costly numerical integration and complicated parameter regularization. The two-stage smoothing-based estimation method (\cite{voit2004decoupling,liang2008parameter}), decouples the system of differential equations into a set of pseudo-regression models.  This method avoids numerically solving the differential equations directly and does not require the initial or boundary conditions of the state variables. More importantly, it allows us to perform model selection and parameter estimation for one equation at a time, which significantly reduces the computational cost.



\begin{enumerate}

   \item Step 1: Obtain the estimates of the temporal gene response modules $M_{q,j}$ and their derivatives $\frac{dM_{q,j}}{dt}$ using the smoothing splines estimates obtained in step one.
   \item Step 2: We plug the estimated mean expression curves $\hat{M}_{q,j}$ and their derivatives $\frac{d\hat{M}_{q,j}}{dt}$ into the ODE model to obtain the following set of pseudo linear regression models.

    \begin{equation} \frac{d\hat{M}_{q,j}}{dt} = \beta_{0,q,j} + \sum_{p=1}^{P} \beta_{p,q,j}\hat{M}_{p,j} + \epsilon_{p,j}, \end{equation}
\end{enumerate}

for $q=1,...,Q$. For linear regression models, many penalized methods have been proposed in the regularization framework to conduct variable selection and estimation, such as the least absolute shrinkage and selection operator (LASSO) \citep{tibshirani1996regression}, smoothly clipped absolute deviation (SCAD) (Fan and Li, 2001) and so on. Here we use LASSO regularization to conduct variable selection and estimation using the the follwing following penalized objective function

\begin{equation} \epsilon_{p,j} + \lambda \sum_{p=0}^{P} \| \beta_{p,q,j} \| \end{equation}

where $\sum_{p=0}^{P} \| \beta_{p,q,j} \|$ is the LASSO penalty which shrinks the parameters $\beta_{p,q,j}$ to zero and $\lambda$ is the sparisty parameter which controls the trade-off between minimizing the error of the pseudo linear regression model and requiring the parameters $\beta_{p,q,j}$ to be zero. A large $\lambda$ sets all the parameters $\beta_{p,q,j}$ to zero. A low $\lambda$ minmizes the error of the pseudo linear regression model.

The two-stage smoothing-based estimation method is employed to simplify the computation of the ODE models and also facilitate the variable selection procedure. However, the parameter estimates from the two-stage method are not efficient in terms of estimation accuracy, as there can be considerable approximation error in the estimates of the modules expression curves $\hat{M}_{q,j}$ and their derivatives $\frac{d(\hat{M}_{q,j}}{dt}$ and decoupling the system of differential equations into a set of pseudo-regression models only accounts for the direct effect, that is, how the $q^{th}$ gene response module is regulated by the $p^{th}$ gene response module (q $\rightarrow$ p) and does not account for the indirect effect, that is, how the $r^{th}$ gene response module is regulated by the $q^{th}$ gene response module which in turn regululates the $p^{th}$ gene response module (r $\rightarrow$ q $\rightarrow$ p) which is a fundamental property of any system of differential equations.

To overcome the estimation deficiency and the decoupling effect of the two-stage method, we propose to refine the parameter estimates for the selected ODE model using the nonlinear least squares (NLS) method. The parameter estimates from the two-stage method are used to estimate the sturcture of the estimates only, that is, to inform us which parameters are non-zero.


\subsection{Annotation of the GRMs}
\label{section:david}

The genes in all the GRMs can be annotated using the \textit{Database for Annotation, Visualization and Integrated Discovery (DAVID)} \citep{huang2009systematic,huang2009bioinformatics}. The annotation returned by \textit{DAVID} is composed of the following reports.

\begin{enumerate}
\item Cluster Report
\item Chart Report
\item Table Report   
\end{enumerate}

\section{Results}
\label{section:results}

\subsection{General results from all conditions}
\label{section:general_results}

\par Summary statistics are displayed in Table~\ref{table:summary}.\footnote{Full data can be found in supplementary file \href{Summary.csv}{Summary.csv}.}

\par Frequency of DRGs across all analyzed conditions are listed in Table~\ref{table:frequent_DRGs}.\footnote{Full data can be found in supplementary file \href{Frequency\_of\_DRGs.csv}{Frequency\_of\_DRGs.csv}.}


