

Figure~\ref{fig:smoothexp} shows the smooth expression curves.

Figure~\ref{figure:drgs} shows the expression of the top ranking dynamic response genes identified using the method described in Section~\ref{section:identification_of_drgs}.

Figure~\ref{figure:grms} shows the expression of gene response modules identified using the method described in Section~\ref{section:identification_of_grms}.

Figure~\ref{figure:grmstype} shows the GRMs' mean expression curves grouped into four categories by cluster size. The four categories are single-gene modules (SGM) with only one gene in each cluster, small-size modules (SSM) that contain between 2-10 genes in each cluster, medium-size modules (MSM) that consist of 11-99 genes in each of the clusters and large-size modules (LSM) which contain over 100 genes in each cluster.

Figure~\ref{fig:generegnet} shows the gene regulatory network discovered using the method described in Section~\ref{section:identification_of_grn}.

Graph theorists and network analysts have developed a number of metrics to characterize biological networks \cite{huber2007graphs, lee2004coexpression}. These metrics facilitate drug target identification and insight on potential strategies for treating various diseases.

Table~\ref{table:graphstats} shows the graph statistics of the gene regulatory network in Figure~\ref{fig:generegnet}.\footnote{These graph statistics can be found in supplementary file \href{Step_7/Graph\_Statistics.xls}{Graph\_Statistics.xls}} Table~\ref{table:nodestats} provides an overview of the node statistics in the same network, listing the top and bottom ranking nodes (modules) in each metric.\footnote{The full list of node statistics can be found in supplementary file \href{Step_7/Node\_Statistics.xls}{Node\_Statistics.xls}}

\begin{figure}
\centering
\includegraphics[width=6in]{Step_2/Paper_01.png}
\caption{All genes}
\label{fig:allgenes}
\end{figure}

\begin{figure}
\includegraphics[width=6in]{Step_3/Paper_03.pdf}
\caption{Smooth expression of all genes.}
\label{fig:smoothexp}
\end{figure}

\begin{figure}
\includegraphics[width=6in]{Step_3/Paper_04.png}
\caption{Dynamic response genes.}
\label{figure:drgs}
\end{figure}

\begin{figure}
\includegraphics[width=6in]{Step_3/Paper_05.png}
\caption{Top ranking DRGs.}
\label{figure:drgs}
\end{figure}

\begin{figure}
\includegraphics[width=6in]{Step_4/GRMs_1.pdf}
\caption{Expression of the gene response modules.}
\label{figure:grms}
\end{figure}

\begin{figure}
\includegraphics[width=6in]{Step_4/GRMs.pdf}
\caption{Mean curves of gene response modules grouped by cluster size.}
\label{figure:grmstype}
\end{figure}

\begin{figure}
\includegraphics[width=6in]{Step_6/Network_plot_MATLAB.pdf}
\caption{Gene regulatory network}
\label{fig:generegnet}
\end{figure}


\subsection{Annotation of the GRMs}
\label{section:results2}

The genes in all the GRMs can be annotated using the \textit{Database for Annotation, Visualization and Integrated Discovery (DAVID)} \cite{huang2009systematic,huang2009bioinformatics}. The annotation returned by the pipeline is composed of the following reports.

\begin{enumerate}
\setlength{\itemsep}{-1ex}
   \item Table Report
   \item Chart Report
\end{enumerate}
